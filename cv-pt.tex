%!TEX TS-program = xelatex
\documentclass[]{friggeri-cv}
\addbibresource{bibliography.bib}

\usepackage{tikz}
\usepackage{multicol}

\begin{document}
\header{André}{Santos}
       {MSc Student \& Ruby on Rails Developer}

\begin{aside}
  \includegraphics[height=100pt]{foto}
  \section{Páginas e Contactos}
    \href{mailito:andreccdr@gmail.com}{andreccdr@gmail.com}
    \href{http://andre-santos.pt}{andre-santos.pt}
    ~
    \href{http://pt.linkedin.com/in/62gerente/}{linkedin/62gerente}
    \href{https://github.com/62Gerente}{github/62Gerente}
    \href{https://twitter.com/62gerente}{twitter/62gerente}
   \section{Idiomas}
    Português
    Inglês Intermédio
  \section{Principais Conhecimentos}
    C, Java
    Ruby, Perl
    Ruby on Rails
    Bases de Dados
    CSS \& HTML
    Git, Bash, \LaTeX
\end{aside}

\section{Sobre}

\textbf{Nome Completo} André Augusto Costa Santos 

\begin{multicols}{3}
\textbf{Morada} Rua S. João nº103  

\textbf{Código Postal} 4760 721 

\textbf{Localidade} Ribeirão
\end{multicols}

\begin{multicols}{3}
\textbf{Telemóvel} 919200815  

\textbf{Email} \href{mailito:andreccdr@gmail.com}{andreccdr@gmail.com}

\textbf{Sexo} Masculino
\end{multicols}

\begin{multicols}{2}
\textbf{Data de Nascimento} 29 de Outubro de 1992 

\textbf{Nacionalidade} Portuguesa
\end{multicols}

\section{Interesses}

Desenvolvimento Ágil de Software, Software Open Source, Novas Tecnologias, Desenvolvimento Web, Desenvolvimento para Dispositivos 
Móveis, Processamento de Linguagens, Administração de Base de Dados

\section{Educação}

\begin{entrylist}
  \entry
    {2013-2015}
    {Mestrado em Engenharia Informática}
    {Universidade do Minho}
    {{\bf Áreas de especificação}\\
    Engenharia de Aplicações\\
    Engenharia de Linguagens}
  \entry
    {2010-2013}
    {Licenciatura em Engenharia Informática}
    {Universidade do Minho}
    {{\bf Principais competências adquiridas} \\ 
    Programação Funcional, Programação Imperativa, 
    Sistemas de Computação, Algoritmos e Complexidade, 
    Arquitetura de Computadores, Sistemas Operativos, 
    Programação Orientada aos Objetos, Bases de Dados,
    Sistemas Distribuídos, Desenvolvimento de Sistemas de Software,
    Redes de Computadores, Computação Gráfica,
    Comunicações por Computador, Processamento de Linguagens
    \\ 15,38 Valores (0-20)}
  \entry
    {2007–2010}
    {Ensino Secundário}
    {Escola Secundária D. Sancho I}
    {Ciências e Tecnologias \\ 15,00 Valores (0-20)}
\end{entrylist}
\newpage
\section{Conhecimentos}

\begin{entrylist}
  \entry
    {}
    {Ruby on Rails}
    {Experiente}
    {Rspec, Cucumber, Devise, Paperclip, Turbolinks, Capybara, Whenever, OmniAuth, Shoulda, factory\_girl, Capistrano, Passenger}
  \entry
    {}
    {Java}
    {Experiente}
    {JFC/Swing, JDBC, I/O, Sockets, Networking, Concurrency}
  \entry
    {}
    {C}
    {Experiente}
    {GLib, I/O, System Calls, Process Communication, Process Control}
  \entry
    {}
    {Perl}
    {Proficiente}
    {Modules, Object Oriented, Testing}
  \entry
    {}
    {Ruby}
    {Proficiente}
    {Blocks, Procs, Dynamic Classes \& Methods}
  \entry
    {}
    {Desenvolvimento web}
    {Experiente}
    {Ruby on Rails, HTML, Slim, CSS, SASS, jQuery, Twitter Bootstrap, AJAX, PHP, ASP.NET, JavaScript, CoffeeScript}
  \entry
    {}
    {Outros}
    {}
    {Database Administration, Database Design, Python, Git, \LaTeX, Haskell, Bash, XML, XSD, XSLT, DTD, UML, OpenGL, Wordpress, MySQL, PostgreSQL, SQLite, Vim, Sublime, Unix \& Linux}
\end{entrylist}

\section{Projetos Tecnológicos}

\begin{entrylist}
  \entry
    {RoRails}
    {Geshora}
    {\href{http://geshora.pt}{Geshora.pt}} 
    {Sistema de gestão de horários e tacógrafos}
  \entry
    {ASP.NET}
    {ArqueoDB}
    {\href{https://github.com/62Gerente/ArqueoDB}{github/62Gerente/ArqueoDB}} 
    {Comunidade de partilha de conhecimento sobre locais, achados e factos arqueológicos.}
\end{entrylist}

\section{Atividades e Participações}

\textbf{\Large \href{http://www.cesium.di.uminho.pt/}{CeSIUM}} 
\\Centro de Estudantes de Engenharia Informática da Universidade do Minho

\begin{entrylist}
  \entry
    {2013-2014}
    {Vice-Presidente CeSIUM}
    {\href{http:www.cesium.di.uminho.pt}{www.cesium.di.uminho.pt}} 
    {Centro de Estudantes de Engenharia Informática da Universidade do Minho}
  \entry
    {2012-2013}
    {Vice-Diretor CAOS}
    {\href{http://caos.di.uminho.pt/}{caos.di.uminho.pt}} 
    {Centro de Apoio ao Open Source}
  \entry
    {2010-2012}
    {Membro CAOS}
    {\href{http://caos.di.uminho.pt/}{caos.di.uminho.pt}} 
    {Centro de Apoio ao Open Source}
\end{entrylist}

\textbf{\Large \href{http://www.yme.pt/‎}{Young Minho Enterprise}} 

\begin{entrylist}
  \entry
    {2013-2014}
    {Membro do Departamento Comercial}
    {\href{http://www.yme.pt/}{yme.pt}} 
    {}
\end{entrylist}

\textbf{\Large MIUP e Codebits} 

\begin{entrylist}
  \entry
    {2012}
    {Participação MIUP 2012}
    {\href{http://miup2012.dcc.fc.up.pt/}{miup2012.dcc.fc.up.pt}} 
    {Maratona Inter-Universitária de Programação}
  \entry
    {2012}
    {Participação Codebits VI}
    {\href{https://codebits.eu/}{codebits.eu}} 
    {Maior evento de tecnologia em Portugal}
\end{entrylist}

\textbf{\Large Desporto} 

\begin{entrylist}
  \entry
    {2007-2011}
    {Atletismo}
    {Clube de Cultura e Desporto de Ribeirão} 
    {Lançamento do disco}
\end{entrylist}

\section{Eventos}
\begin{entrylist}
  \entry
    {Organizador}
    {Coderdojo Minho}
    {\href{https://www.facebook.com/CoderdojoMinho}{facebook/CoderdojoMinho}} 
    {Movimento aberto e sem fins lucrativos para ensinar jovens dos 7 aos 17 a programar.}
  \entry
    {Organizador}
    {CodeWeek@DI}
    {\href{http://www.cesium.di.uminho.pt/2013/11/23/codeweek-at-di}{cesium/codeweek-at-di}} 
    {Semana inserida no CodeWeek@EU, repleta de partilha de ideias e desafios para os alunos de informática da Universidade do Minho.}
  \entry
    {Membro}
    {Minho.RB}
    {\href{http://www.meetup.com/Minho-rb/}{meetup/Minho-rb}} 
    {Encontro informal de aficionados de Ruby para partilhar experiências sobre Ruby, Rails e outras tecnologias relacionadas.}
\end{entrylist}

\section{Cursos}

\begin{entrylist}
  \entry
    {2013}
    {Creators School}
    {\href{http://cs.groupbuddies.com/}{cs.groupbuddies.com}} 
    {Business Development, Product Development, Web Design and Web Development}
  \entry
    {2012}
    {Curso de Inglês}
    {\href{http://www.babelium.uminho.pt/}{babelium.uminho.pt/}} 
    {Nível B1}
\end{entrylist}


\end{document}
