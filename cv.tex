%!TEX TS-program = xelatex
\documentclass[]{friggeri-cv}
\addbibresource{bibliography.bib}

\usepackage{tikz}
\usepackage{multicol}

\begin{document}
\header{André}{Santos}
       {Licenciado em Engenharia Informática}

\begin{aside}
  \includegraphics[height=100pt]{foto}
  \section{Páginas e Contactos}
    \href{mailito:andreccdr@gmail.com}{andreccdr@gmail.com}
    \href{http://andre-santos.pt}{andre-santos.pt}
    ~
    \href{http://pt.linkedin.com/in/62gerente/}{linkedin/62gerente}
    \href{https://github.com/62Gerente}{github/62Gerente}
    \href{https://www.facebook.com/andreccdr}{facebook/andreccdr}
    \href{https://twitter.com/62gerente}{twitter/62gerente}
    \href{http://www.codeschool.com/users/gerente}{codeschool/gerente}
    \href{https://codebits.eu/gerente}{codebits/gerente}
   \section{Idiomas}
    Português
    Inglês Intermédio
  \section{Principais Conhecimentos}
    C, Java, Ruby
    Ruby on Rails
    ASP.NET
    Haskell
    CSS \& HTML
    Git, Bash, \LaTeX
\end{aside}

\section{Sobre}

\textbf{Nome Completo} André Augusto Costa Santos

\begin{multicols}{3}
\textbf{Morada} Rua S. João nº103  

\textbf{Código Postal} 4760 721 

\textbf{Localidade} Ribeirão
\end{multicols}

\begin{multicols}{3}
\textbf{Telemóvel} 919200815  

\textbf{Email} \href{mailito:andreccdr@gmail.com}{andreccdr@gmail.com}

\textbf{Sexo} Masculino
\end{multicols}

\begin{multicols}{2}
\textbf{Data de Nascimento} 29 de Outubro de 1992 

\textbf{Nacionalidade} Portuguesa
\end{multicols}

\section{Interesses}

Desenvolvimento Ágil de Software, Software Open Source, Novas Tecnologias, Desenvolvimento Web, Desenvolvimento para Dispositivos 
Móveis, Sistemas de Suporte à Decisão, Sistemas Distribuídos, Processamento de Linguagens 

\section{Educação}

\begin{entrylist}
  \entry
    {2010-2013}
    {Licenciatura em Engenharia Informática}
    {Universidade do Minho}
    {{\bf Principais competências adquiridas} \\ 
    Programação Funcional, Programação Imperativa, 
    Sistemas de Computação, Algoritmos e Complexidade, 
    Arquitetura de Computadores, Sistemas Operativos, 
    Programação Orientada aos Objetos, Bases de Dados,
    Sistemas Distribuídos, Desenvolvimento de Sistemas de Software,
    Redes de Computadores, Computação Gráfica,
    Comunicações por Computador, Processamento de Linguagens
    \\ 15,38 Valores (0-20)}
  \entry
    {2012}
    {Curso de Inglês}
    {BabeliUM}
    {Nível B1}
  \entry
    {2007–2010}
    {Ensino Secundário}
    {Escola Secundária D. Sancho I}
    {Ciências e Tecnologias \\ 15,00 Valores (0-20)}
\end{entrylist}
\newpage
\section{Conhecimentos}

\begin{entrylist}
  \entry
    {}
    {Ruby on Rails}
    {Experiente}
    {Rspec, Cucumber, Devise, Paperclip, Turbolinks, Capybara, Whenever, OmniAuth, Shoulda, factory\_girl, Capistrano, Passenger}
  \entry
    {}
    {Java}
    {Experiente}
    {JFC/Swing, JDBC, I/O, Sockets, Networking, Concurrency}
  \entry
    {}
    {C}
    {Experiente}
    {GLib, I/O, System Calls, Process Communication, Process Control}
  \entry
    {}
    {Ruby}
    {Proficiente}
    {Blocks, Procs, Dynamic Classes \& Methods}
  \entry
    {}
    {Desenvolvimento web}
    {Experiente}
    {Ruby on Rails, HTML, CSS, jQuery, Twitter Bootstrap, AJAX, ASP.NET}
  \entry
    {}
    {Outros}
    {}
    {Ruby, Git, \LaTeX, Haskell, Bash, UML, OpenGL, Wordpress, Oracle, MySQL, Vim, Sublime}
\end{entrylist}

\section{Projetos}

\begin{entrylist}
  \entry
    {RoRails}
    {Geshora}
    {\href{http://geshora.pt}{Geshora.pt}} 
    {Sistema de gestão de horários e tacógrafos}
  \entry
    {ASP.NET}
    {ArqueoDB}
    {\href{https://github.com/62Gerente/ArqueoDB}{github/62Gerente/ArqueoDB}} 
    {Comunidade de partilha de conhecimento sobre locais, achados e factos arqueológicos.}
\end{entrylist}

\section{Atividades e Participações}

\textbf{\Large \href{http://www.cesium.di.uminho.pt/}{CeSIUM}} 
\\Centro de Estudantes de Engenharia Informática da Universidade do Minho

\begin{entrylist}
  \entry
    {2012-2013}
    {Vice-Diretor CAOS}
    {\href{http://caos.di.uminho.pt/}{caos.di.uminho.pt}} 
    {Centro de Apoio ao Open Source}
  \entry
    {2010-2012}
    {Membro CAOS}
    {\href{http://caos.di.uminho.pt/}{caos.di.uminho.pt}} 
    {Centro de Apoio ao Open Source}
\end{entrylist}

\textbf{\Large MIUP e Codebits} 

\begin{entrylist}
  \entry
    {2012}
    {Participação MIUP 2012}
    {\href{http://miup2012.dcc.fc.up.pt/}{miup2012.dcc.fc.up.pt}} 
    {Maratona Inter-Universitária de Programação}
  \entry
    {2012}
    {Participação Codebits VI}
    {\href{https://codebits.eu/}{codebits.eu}} 
    {Maior evento de tecnologia em Portugal}
\end{entrylist}

\textbf{\Large Desporto} 

\begin{entrylist}
  \entry
    {2007-2011}
    {Atletismo}
    {Clube de Cultura e Desporto de Ribeirão} 
    {Lançamento do disco}
\end{entrylist}

\end{document}
