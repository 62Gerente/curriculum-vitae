%!TEX TS-program = xelatex
\documentclass[]{friggeri-cv}
\addbibresource{bibliography.bib}

\usepackage{tikz}
\usepackage{multicol}

\begin{document}
\header{André}{Santos}
       {Estudante de Engenharia Informática}


% In the aside, each new line forces a line break
\begin{aside}
  \includegraphics[height=100pt]{foto}
  \section{Páginas e Contactos}
    \href{mailito:andreccdr@gmail.com}{andreccdr@gmail.com}
    \href{http://about.me/andresantos62}{about/andresantos62}
    ~
    \href{https://github.com/62Gerente}{github/62Gerente}
    \href{https://www.facebook.com/andreccdr}{facebook/andreccdr}
    \href{https://twitter.com/62gerente}{twitter/62gerente}
    \href{http://www.codeschool.com/users/gerente}{codeschool/gerente}
    \href{https://codebits.eu/gerente}{codebits/gerente}
   \section{Idiomas}
    Português
    Inglês Intermédio
  \section{Principais Conhecimentos}
    C
    Java
    Haskell
    CSS \& HTML
    Git, Bash, \LaTeX
\end{aside}

\section{Sobre}

\textbf{Nome Completo} André Augusto Costa Santos

\begin{multicols}{3}
\textbf{Morada} Rua S. João nº103  

\textbf{Código Postal} 4760 721 

\textbf{Localidade} Ribeirão
\end{multicols}

\begin{multicols}{3}
\textbf{Telemóvel} 919200815  

\textbf{Email} \href{mailito:andreccdr@gmail.com}{andreccdr@gmail.com}

\textbf{Sexo} Masculino
\end{multicols}

\begin{multicols}{2}
\textbf{Data de Nascimento} 29 de Outubro de 1992 

\textbf{Nacionalidade} Portuguesa
\end{multicols}

\section{Interesses}

Software Open Source, Novas Tecnologias, Desenvolvimento Web, Desenvolvimento para dispositivos 
móveis, Bases de Dados, Sistemas Distribuídos, Processamento de Linguagens

\section{Educação}

\begin{entrylist}
  \entry
    {desde 2010}
    {Licenciatura em Engenharia Informática}
    {Universidade do Minho}
    {{\bf Principais cadeiras concluídas} \\ 
    Programação Funcional, Programação Imperativa, 
    Sistemas de Computação, Algoritmos e Complexidade, 
    Arquitetura de Computadores, Sistemas Operativos, 
    Programação Orientada aos Objetos, Bases de Dados,
    Sistemas Distribuídos, Desenvolvimento de Sistemas de Software,
    Redes de Computadores  \\ 15,33 Valores (0-20)}
  \entry
    {2012}
    {Curso de Inglês}
    {BabeliUM}
    {Nível B1}
  \entry
    {2007–2010}
    {Ensino Secundário}
    {Escola Secundária D. Sancho I}
    {Ciências e Tecnologias \\ 15,00 Valores (0-20)}
\end{entrylist}

\section{Conhecimentos}

\textbf{\large Programação Imperativa}

Experiência e conhecimentos sólidos na Linguagem de Programação C, realização de 
vários projectos no âmbito de cadeiras da licenciatura.

\textbf{\large Programação Orientada a Objectos}

Experiência e conhecimentos na Linguagem de Programação Java, realização de 
alguns projectos no âmbito de cadeiras da licenciatura. Em iniciação a C++,C\# e 
Ruby.

\textbf{\large Desenvolvimento Web}

Conhecimentos de HTML e CSS, actualmente a desenvolver uma aplicação web 
utilizando a plataforma da Microsoft ASP.NET.
Experiência com Twitter Bootstrap e com o sistema de gestão de conteúdo Wordpress.

\newpage
\textbf{\large Sistemas UNIX}

Experiência com sistemas Linux/Unix, ganha com o uso diário de sistemas deste tipo.
Conhecimentos de Bash, principalmente adquiridos no desenvolvimento de pequenos scripts 
para facilitar o uso diário da mesma e no âmbito de algumas cadeiras da licenciatura.

\textbf{\large Outros Conhecimentos}

\textbf{Conhecimentos Sólidos}

Haskell, \LaTeX, Git, Oracle, UML

\textbf{Em Iniciação}

Rails, jQuery, PHP, JavaScript
\\
\section{Actividades e Participações}

\textbf{\LARGE \href{http://www.cesium.di.uminho.pt/}{CeSIUM}} 
\\Centro de Estudantes de Engenharia Informática da Universidade do Minho

\begin{entrylist}
  \entry
    {2012-2013}
    {Vice-Director CAOS}
    {\href{http://caos.di.uminho.pt/}{caos.di.uminho.pt}} 
    {Centro de Apoio ao Open Source}
\end{entrylist}

\textbf{\LARGE MIUP e Codebits} 

\begin{entrylist}
  \entry
    {2012}
    {Participação MIUP 2012}
    {\href{http://miup2012.dcc.fc.up.pt/}{miup2012.dcc.fc.up.pt}} 
    {Maratona Inter-Universitária de Programação}
  \entry
    {2012}
    {Participação Codebits VI}
    {\href{https://codebits.eu/}{codebits.eu}} 
    {Maior evento de tecnologia em Portugal}
\end{entrylist}

\textbf{\LARGE Desporto} 

\begin{entrylist}
  \entry
    {2007-2011}
    {Atletismo}
    {Clube de Cultura e Desporto de Ribeirão} 
    {Lançamento do disco}
\end{entrylist}

\end{document}
